\section{Introduction}
\label{sec:introduction}

Massive stars are the cornerstone of many fields of astrophysics.
The ionizing radiation emitted at the surfaces of these stars serves as an important feedback mechanism which regulates star formation and the ISM structure \citep{lancaster_etal_2021}, and this radiation was important in the reionization of the early universe \citep{bromm_larson_2004}.
Massive stars end their lives in dynamic explosions which produce supernovae and observable high-energy transients \citep{heger_2003}.
These explosions produce compact remnants such as black holes and neutron stars and ground-based gravitational wave interferometers like LIGO have recently begun to probe the mass distribution of these objects \citep{abbott_etal_2018}.
The complex evolutionary history of massive stars determines what type of explosion these stars end their lives in and what type of remnant they leave behind \citep{farmer_etal_2016}.
A detailed understanding of the internal structure and evolution of massive stars is therefore desired.


\begin{enumerate}
\item Our best tool for probing the insides of massive stars is Asteroseismology.
    This is going really well for SPB and $\gamma$ Dor stars.
\item Recently, stochastic low-frequency variability has been seen on even hotter, more massive stars, and there is a current debate about whether that signal is gravity waves driven by core convection or something else.
\item In order to resolve this debate, and to compare simulations of waves in massive stars to observations of massive stars, a framework is required for computing the spectra that would be observed at the surface of theoretical ``stars'' given the flow fields in simulations.
\item In this paper, we build upon the framework laid out in \citet{lecoanet_etal_2019,lecoanet_etal_2021}. 
    We demonstrate that the surface signal of linear waves can be predicted given the linear eigenvalues and a power-law description of the wave luminosity produced by the convection.
\end{enumerate}
